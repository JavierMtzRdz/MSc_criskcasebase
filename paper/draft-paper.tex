\documentclass[AMA,Times1COL]{WileyNJDv5} %STIX1COL,STIX2COL,STIXSMALL

\articletype{Article Type}%

\received{Date Month Year}
\revised{Date Month Year}
\accepted{Date Month Year}
\journal{Journal}
\volume{00}
\copyyear{2023}
\startpage{1}

\raggedbottom

%%\bibliography{paper/refs/competing-risk}

\usepackage{dsfont} 
\begin{document}

\title{cbSCRIP: Casebase Sampling for Sparse Competing Risk Prediction}

\author[1]{Author One}

\author[2,3]{Author Two}

\author[3]{Author Three}

\authormark{TAYLOR \textsc{et al.}}
\titlemark{PLEASE INSERT YOUR ARTICLE TITLE HERE}

\address[1]{\orgdiv{Department Name}, \orgname{Institution Name}, \orgaddress{\state{State Name}, \country{Country Name}}}

\address[2]{\orgdiv{Department Name}, \orgname{Institution Name}, \orgaddress{\state{State Name}, \country{Country Name}}}

\address[3]{\orgdiv{Department Name}, \orgname{Institution Name}, \orgaddress{\state{State Name}, \country{Country Name}}}

\corres{Corresponding author Mark Taylor, This is sample corresponding address. \email{authorone@gmail.com}}

\presentaddress{This is sample for present address text this is sample for present address text.}

%\fundingInfo{Text}
%\JELinfo{ejlje}

\abstract[Abstract]{In biomedical studies, quantifying the association between prognostic genes and markers with time-to-event outcomes is crucial for predicting a patient's disease risk based on their specific covariate profile. Modeling competing risks is necessary, as patients may face multiple mutually exclusive events, such as death from different causes. However, current methods for competing risks analysis often yield coefficient estimates that are difficult to interpret, making it challenging to connect them to event rates. Additionally, the high dimensionality of genomic data, where the number of variables exceeds the number of subjects, presents a significant challenge.

In this work, we propose a novel approach using an elastic-net penalized multinomial model within a case-based sampling framework to analyze competing risks survival data. We also introduce a two-step method called the de-biased case-base to improve the predictive performance regarding disease risk. Through a comprehensive simulation study that replicates biomedical data, we show that the casebase method is effective in variable selection and survival prediction, particularly in scenarios involving non-proportional hazards. Moreover, we highlight the flexibility of this approach in generating smooth-time incidence curves, which significantly improve the accuracy of patient risk estimation assessments.}

\keywords{keyword1, keyword2, keyword3, keyword4}

\maketitle

\renewcommand\thefootnote{}
\footnotetext{\textbf{Abbreviations:} ANA, anti-nuclear antibodies; APC, antigen-presenting cells; IRF, interferon regulatory factor.}

\renewcommand\thefootnote{\fnsymbol{footnote}}
\setcounter{footnote}{1}

\section{INTRODUCTION}\label{sec1}

In modern biomarker studies, patients are often tracked over time while recording high-dimensional molecular, imaging, or clinical data. A main aim of such studies is to identify risk factors of a primary event of interest. However, alternative mutually exclusive events often exist. For example, among HIV-infected hemophiliacs, AIDS has historically been the leading cause of death, while deaths from non-AIDS-related conditions, such as liver disease and hemorrhagic complications, have emerged as important competing risks.\cite{AmoPerez-HoyosMoreno:2006} In classical survival analysis, outcomes other than the event of interest, such as loss to follow-up or end of the study, are treated as censored, under the assumption that censored observations carry no information about the future risk of the primary event. However, competing events that prevent the occurrence of the primary event may violate this assumption. Treating such events as censored can lead to biased estimates of the cumulative incidence of the event of interest. Alternatively, primary and competing events can be combined into a composite endpoint. However, this approach masks differences in clinical relevance and prevents attribution of covariate effects to specific outcomes. In these settings, competing risks models are essential to properly quantify the absolute risk and relative contribution of each event type. 

The Cox proportional hazards (CPH) model is widely used in survival analysis to examine the association between covariates and the hazard of a single event of interest, without imposing a specific parametric form on the baseline hazard.\cite{Cox:1972} In the presence of competing risks, extensions of the CPH model typically involve fitting separate cause-specific hazard models for each event type.\cite{PrenticeKalbfleischPeterson:1978} Although this allows for estimation of covariate effects on each cause-specific hazard, the cumulative incidence function for the primary event still depends on the cause-specific hazards of all other events. To address this limitation, Fine and Gray introduced subdistribution hazards to model only the hazard associated with the cumulative incidence for the event of interest.\cite{FineGray:1999} While the Fine–Gray model estimates covariate effects on the subdistribution hazard of the primary event and the cumulative incidence function, it does not provide the effects on the cause-specific hazard among all individuals still at risk. Moreover, the estimated cumulative incidence probabilities may exceed 1 under certain covariate combinations, limiting their clinical applicability. See Saadati et al. \cite{SaadatiBeyersmannKopp-Schneider:2018} for a detailed discussion of these models.

Another important limitation of many existing approaches is that they often rely on the proportional hazards assumption to estimate covariate effects using semi-parametric models. However, the flexibility of these semi-parametric models come at the cost of requiring separate estimation of the baseline hazard, resulting in stepwise survival estimates that can be difficult to interpret. Combining a case-base sampling framework with theoretical results from counting processes, Hanley and Miettinen proposed a fully parametric estimation approach based on logistic regression that produces smooth cumulative incidence functions over time.\cite{HanleyMiettinen:2009, Saarela:2016} This approach can be naturally extended to competing risks settings using a multinomial regression with an offset term, as implemented in the Casebase R package.\cite{BhatnagarTurgeonIslam:2022} Under this alternative paradigm, hazard functions are estimated using familiar tools from generalized linear modeling, providing smooth estimates over continuous time while preserving interpretability and flexibility.

As high-dimensional data become more prevalent in competing risks studies, regularized methods have been proposed to select relevant variables associated with cause-specific hazards and cumulative incidence functions. A growing body of work has focused on extending penalized regression techniques, such as the LASSO, Elastic Net, and boosting, to the cause-specific hazard models,\cite{SaadatiBeyersmannKopp-Schneider:2018, HouParavatiHou:2018} while others have developed high-dimensional adaptations of the Fine–Gray model for subdistribution hazards.\cite{BinderAllignolSchumacher:2009, HouBradicXu:2019} Despite increasing attention to high-dimensional modeling in competing risks settings, there remains limited work addressing the joint modeling of cause-specific hazards and cumulative incidence functions in high-dimensional settings. Recent reviews of these approaches highlight both methodological advances and remaining challenges.\cite{Monterrubio-GomezConstantine-CookeVallejos:2024, HouXu:2018} 

Most existing approaches focus on only one component of the risk structure and do not readily allow estimation of one from the other, which can limit interpretability and hinder a comprehensive understanding of event dynamics. In this paper, we build on the case-base sampling framework to propose an alternative approach, cbSCRIP, for estimating sparse competing risk models aimed at prediction. Our parametric method, based on regularized multinomial regression, models the case-specific hazards and thus provides smooth estimates of cumulative incidence functions. It also offers a flexible framework for scalable variable selection in high-dimensional settings and the modelling of time-varying covariates. We demonstrate that cbSCRIP exhibits strong variable selection performance in both low- and high-dimensional settings, while effectively predicting cumulative risk. These properties make it a viable and clinically useful option for the analysis of survival data with competing events.

The rest of the paper is organized as follows. Section 2 introduces the case-base framework and its extension to multinomial regression for competing risks, including the debiasing procedure used to recover accurate and smooth cumulative incidence estimates. Section 3 presents results from simulation studies evaluating variable selection, prediction performance, and the quality of cumulative incidence estimation. Section 4 applies the proposed method to the [XXXX] dataset to assess its performance on real-world medical data. Finally, Section 5 discusses the limitations of the approach, potential extensions, and directions for future research.


\section{METHODS}\label{sec2}

\subsection{Case-base Sampling in Competing Risk Analysis}

In revising established ideas and concepts in epidemiology, Miettinen proposed in his 1999 article\cite{Miettinen:1999} and discussed further in \textit{Epidemiology: Quo Vadis},\cite{Miettinen:2004} the replacement of the long-standing concepts of "cohort" and "case-control study" with the alternative "study base" and "case-base" in etiologic study designs. The case-base sampling framework considers a finite sample of person-moments, each defined as a "person" coordinate $i$ at a specific point in time $t$, from the study base, which conceptually consists of an infinite number of such person-moments spanning all observed follow-up experiences. The fundamental idea of case-base sampling is to generate a "case series" by sampling all person-moments when the event occurred, and a "base series" by sampling person-moments when individuals were at risk of the event from the study base.\citep{HanleyMiettinen:2009} This method of representative sampling differs from the risk-set sampling used in Cox regression and aims to capture the entire baseline hazard. 

Extending ideas from counting processes, Hanley and Miettinen showed that, under the case-base sampling framework, the estimation of hazard functions in continuous time can be reformulated as a logistic regression problem.\cite{HanleyMiettinen:2009} An important benefit of this approach is its capacity to directly estimate the hazard function, which then enables the derivation of smooth-in-time survival and cumulative incidence functions. Saarela and Arjas extended this approach to competing risk settings.\cite{SaarelaArjas:2015} They showed that using separate case series for each event type and an efficient sampling procedure for generating the base series in the presence of competing events, the conditional likelihood contributions of each individual has the form of a multinomial regression with an offset parameter.

Beyond the broader philosophical discussion of this alternative paradigm, leveraging the well-established framework of generalized linear models to fit flexible hazard regression models offers a major methodological advantage for competing risk analysis. This methodological shift is particularly well suited to high-dimensional settings.\cite{BhatnagarTurgeonIslam:2022} 

\subsection{Notation and Multinomial Parameterization}

To define notation and provide essential background, in this Section, we present some of the main theoretical results on the combined framework of case-base sampling and counting processes. We refer to previous contributions in this topic for additional details (e.g., \cite{SaarelaArjas:2015, AalenBorganGjessing:2008, ArjasHaara:1987}).

Counting processes are stochastic processes used to model the occurrence of events over time. For simplicity, we focus on competing risk studies with two causes of death. In this context, a counting process $N_{ij}(t) \in \{0,1\}$ indicates whether the event type $j$ has occurred for individual $i$ by time $t$, where $i \in \mathcal{C} \equiv \{1, \ldots, n\}$ be the set of individuals who are followed up to time $\tau$, $t \in [0, \tau]$, and $j = 0,1,2$. Specifically, $j = 1$ indicates a primary event of interest, $j = 2$ indicates a competing event, and $j = 0$ indicates type I censoring due to the end of the study (extensions to the general case of independent censoring have been developed).\cite{Saarela:2016} In addition to the outcome event, let $X_i = (X_{i1}, \ldots, X_{ip})$ be a set of covariates with baseline characteristics (extensions to more general settings have also been proposed, e.g., Saarela (2016)\cite{Saarela:2016}).

In stochastic process theory, counting processes are defined relative to the history, which encodes the accumulated information available up to time $t$ for individual $i$. The past is usually modeled as a $\sigma$-algebra of events, which may capture not only the history of the observed process but also external covariates or additional information known at the time. For simplicity, we will refer to this as "past" (e.g., for a similar approach and for additional details).\cite{AalenBorganGjessing:2008, BhatnagarTurgeonIslam:2022, SaarelaArjas:2015}

The framework of counting processes allows event intensities (i.e., hazard functions) to be rigorously defined as conditional expectations given the past. Let $dN_{ij}(t) = 1$ be the differential change of the counting process $N_ij$ in the small time interval $[t, t + dt)$. Assuming that at most one event may occur in such interval, we can define the cause-specific hazard function $\lambda_{ij}(t)$ as the conditional probability that the $i$th individual experiences the event $j$ in $[t, t + dt)$, given the observed information before that interval, divided by the length of the interval,

$$
\lambda_{ij}(t)\,dt = P(dN_{ij}(t) = 1 \mid \text{past}).
$$

In the context of absorbing event states (e.g., death), $dN_{ij}(t)$ can only take values $0$ and $1$, thus, we can also relate the cause-specific hazard function, $\lambda_{ij}(t)$, to the (conditional) expected differential change in the counting process through the following expression,\cite{Saarela:2016} 

$$
\lambda_{ij}(t)\,dt = E(dN_{ij}(t) \mid \text{past}).
$$

If the hazard function $\lambda_{ij}(t, \boldsymbol{\theta}_j)$ is parametrized in terms of $\boldsymbol{\theta}_j$, where $\boldsymbol{\theta}_j \in \mathcal{R}^{p+1}$ is a vector of coefficients, we can estimate these parameters using a maximum likelihood approach. While likelihood-based inference in continuous-time survival models is formally grounded in the relationship between the hazard function and the conditional expectation of the counting process, direct evaluation of the likelihood function involves integrals over time that are computationally intensive. Instead, we can leverage the case–base sampling framework, which conditions on sampled person-moments to derive a more tractable form of the likelihood\cite{SaarelaArjas:2015}. 

In case–base sampling, the case series for the event $j$ consists to all person-moments from the study base at which the outcome event $j$ occur, i.e., $\mathrm{d}N_{ij}(t) = 1$, for $j = {1,2}$. Different sampling mechanisms characterized by an intensity functions $\rho_i(t)$ have been proposed to form a base series. Let $R_i(t) \in \{0, 1, 2, \ldots\}$ be the counting process for the person-moments of individual $i$ in the base case (see Saarela and Arjas (2015) for details on different sampling mechanism\cite{SaarelaArjas:2015}). Thus, the process $Q_i(t) = N_{i1}(t) + N_{i2}(t) + R_i(t)$ counts both the case and base series person–moments contributed by individual $i$.

Under this approach, the likelihood contribution from each sampled person-moment can be written as:

$$
L(\theta) = \prod_{i=1}^n \prod_{t \in [0, \tau]} \left( \frac{\prod_{j=1}^{2}\lambda_{ij}(t; \theta)^{dN_{ij}(t)}}{\rho_i(t) + \sum_{j=1}^{2}\lambda_{ij}(t; \theta)} \right)^{dQ_{ij}(t)}.
$$

This expression is similar to the likelihood function for multinomial regression with an offset term $\log(1/\rho_i(t))$, enabling the use of optimization algorithms developed for generalized linear models to estimate the parameters of the cause-specific hazards. In this paper, we integrate this formulation with regularization techniques and develop an efficient algorithm to scale up competing risk analyses in high-dimensional settings involving large numbers of covariates.

\subsection{Regularization and Optimization}

We can construct a multinomial logistic model for the casebase likelihood. We use $Y_{i}$ to denote a categorical response variable. Let us define $Y_{i}$ to have three levels for the case of two competing events, i.e., {0,1,2}, where the total number of competing events is two. The likelihood function can be written as:
$$\log\left(\frac{\Pr(Y=j|X_i)}{\Pr(Y=0|X_i)}\right)$$ where class 0, i.e., the censored individuals, serves as the reference class. The likelihood is defined for a set of covariates $X_{i}$ for individual $i$.

In terms of optimizing a penalized likelihood, the glmnet package has implemented fast algorithms for several generalized linear models, including multinomial regression with the elastic-net family penalty. However, the glmnet package uses a symmetric parameterization of the multinomial model. This parameterization estimates the relative differences between classes rather than the absolute probabilities for each class, resulting in the constant offset term not being fitted. Since the case base approach relies on the constant offset of the intensity function, we need a penalized model using the multinomial logistic parameterization. Developing a function to fit and tune this penalized model is a key contribution of this work.

The penalized multinomial logistic regression is fitted using accelerated stochastic variance reduced gradient descent (ASVRG), as this algorithm shows fast convergence for high-dimensional datasets with a larger number of predictors than observations ($p> n$). This accelerated method combines SVRG's variance reduction with momentum to speed up convergence.\cite{DriggsEhrhardtSchonlieb:2022} Like SVRG, this algorithm begins with a full-batch gradient computation. However, instead of relying solely on stochastic updates, it employs a sequence of extrapolated points, constructed as a weighted combination of the current and previous iterations.

Here, we denote $l(\beta)$ as the joint likelihood across all individuals $i=1,...,n$ and causes $j=1,2$, i.e., $\sum_{i=1}^{n}\sum_{j=1}^{2}\log(\frac{\text{Pr}(Y_{i}=j|X_{i})}{\text{Pr}(Y_{i}=0|X_{i})})$. For covariates $k\in\{1,...,p\}$, we can estimate the coefficient matrix $\boldsymbol{\beta}$ as
$$\hat{\beta}=\text{arg max}_{\beta}\{l(\beta)+\lambda\sum_{k=1}^{p}w_{k}((\frac{1-\phi}{2})\sum_{j=1}^{2}\beta_{kj}^{2}+\phi\sum_{j=1}^{2}|\beta_{kj}|)\}.$$

$\phi$ is the mixing parameter between the LASSO and ridge penalties. Setting $\phi=1$ and $\phi=0$ corresponds to the LASSO and ridge penalties, respectively. $w_{k}$ represents the penalty factor for the $k^{th}$ covariate, allowing parameters to be penalized differently. In this work, the penalty factor for the intercept and for time is set to 0, so they are unpenalized and always included in the model. Solving for the coefficients $\boldsymbol{\beta}$ will produce a matrix of size $(p+1)\times2$, where the $j^{th}$ column corresponds to the coefficients for cause j, for $j=1,2$.

A cross-validation function was also implemented to fine-tune the shrinkage parameter $\lambda$. The tuning relies on the multinomial deviance as the measure of model goodness-of-fit, shown below:
$$-2\sum_{i=1}^{n}\sum_{j=1}^{2}\log(\frac{\text{Pr}(Y_{i}=j|X_{i})}{\text{Pr}(Y_{i}=0|X_{i})}).$$

\subsection{De-biasing Step}

Pellentesque habitant morbi tristique senectus et netus et malesuada fames ac turpis egestas. Donec odio elit, dictum
in, hendrerit sit amet, egestas sed, leo. Praesent feugiat sapien aliquet odio. Integer vitae justo. Aliquam vestibulum
fringilla lorem. Sed neque lectus, consectetuer at, consectetuer sed, eleifend ac, lectus. Nulla facilisi. Pellentesque
eget lectus. Proin eu metus. Sed porttitor. In hac habitasse platea dictumst. Suspendisse eu lectus. Ut mi mi, lacinia
sit amet, placerat et, mollis vitae, dui. Sed ante tellus, tristique ut, iaculis eu, malesuada ac, dui. Mauris nibh leo,
facilisis non, adipiscing quis, ultrices a, dui.


\section{SIMULATION STUDIES}\label{sec3}

\subsection{kjlj}

Data are simulated from a $K=2$ competing risks proportional hazards model. The cause-specific hazard for cause for individual $i$ with covariates $X_i = (X_{i1}, \dots, X_{ip})$ at time $t$ is $\lambda_k(t | X_i) = \lambda_{0k}(t) \exp(X_i^T \beta_k)$. The baseline hazards $\lambda_{0k}(t)$ follow a Weibull distribution $\lambda_{0k}(t) = h_k \gamma_k t^{\gamma_k - 1}$, with parameters $(h_1, \gamma_1)=(0.55, 1.5)$ and $(h_2, \gamma_2)=(0.35, 1.5)$.

The coefficient vectors $\beta_1, \beta_2 \in \mathbb{R}^p$ are sparse, with non-zero effects predominantly within the first 18 predictors. For cause 1, the coefficients for $X_1, \dots, X_{18}$ are set as $(1, 1, 1, 1, 1, 1,$ $0.5, -0.5, 0.5, -0.5, 0.5, -0.5$ $1, 1, 1, 1, 1, 1)$, and 0 for $j > 18$. For cause 2 ($\beta_2$), the coefficients for $X_1, \dots, X_{24}$ are set as $(0, 0, 0, 0, 0, 0,$ $0.5, -0.5, 0.5, -0.5, 0.5,$ $-0.5, 0, 0, 0, 0, 0, 0,$ $1, 1, 1, 1, 1, 1)$, and 0 for $j > 24$.

Event times $T_i$ and causes $C_i$ are generated by simulating potential failure times $T_{ik}$ from $\lambda_k(t|X_i)$ and setting $T_i = \min(T_{i1}, T_{i2})$ with $C_i$ being the index $k$ for which $T_{ik} = T_i$. Independent censoring times $T_{cens, i}$ are generated based on an overall rate of 0.05. Observed data consist of $(ftime_i, fstatus_i)$, where $ftime_i = \min(T_i, T_{cens, i})$ and the status $fstatus_i = C_i \cdot \mathds{1}(T_i \le T_{cens, i})$ (with $fstatus_i=0$ indicating censoring).


\subsection{Second level head}

Etiam euismod. Fusce facilisis lacinia dui. Suspendisse potenti. In mi erat, cursus id, nonummy sed, ullamcorper
eget, sapien. Praesent pretium, magna in eleifend egestas, pede pede pretium lorem, quis consectetuer tortor sapien
facilisis magna. Mauris quis magna varius nulla scelerisque imperdiet. Aliquam non quam. Aliquam porttitor quam
a lacus. Praesent vel arcu ut tortor cursus volutpat. In vitae pede quis diam bibendum placerat. Fusce elementum
convallis neque. Sed dolor orci, scelerisque ac, dapibus nec, ultricies ut, mi. Duis nec dui quis leo sagittis commodo.
Nulla non mauris vitae wisi posuere convallis. Sed eu nulla nec eros scelerisque pharetra. Nullam varius. Etiam
dignissim elementum metus. Vestibulum faucibus, metus sit amet mattis rhoncus, sapien dui laoreet odio, nec ultricies
nibh augue a enim. Fusce in ligula. Quisque at magna et nulla commodo consequat. Proin accumsan imperdiet sem.
Nunc porta. Donec feugiat mi at justo. Phasellus facilisis ipsum quis ante. In ac elit eget ipsum pharetra faucibus.
Maecenas viverra nulla in massa (Table~\ref{tab2}).

\begin{definition}
Example definition text. Example definition text. Example definition text. Example definition text. Example definition text. Example definition text. Example definition text. Example definition text. Example definition text. Example definition text. Example definition text.
\end{definition}

Sed commodo posuere pede. Mauris ut est. Ut quis purus. Sed ac odio. Sed vehicula hendrerit sem. Duis non
odio. Morbi ut dui. Sed accumsan risus eget odio. In hac habitasse platea dictumst. Pellentesque non elit. Fusce
sed justo eu urna porta tincidunt. Mauris felis odio, sollicitudin sed, volutpat a, ornare ac, erat. Morbi quis dolor.
Donec pellentesque, erat ac sagittis semper, nunc dui lobortis purus, quis congue purus metus ultricies tellus. Proin
et quam. Class aptent taciti sociosqu ad litora torquent per conubia nostra, per inceptos hymenaeos. Praesent sapien
turpis, fermentum vel, eleifend faucibus, vehicula eu, lacus.


\begin{proof}
Example for proof text. Example for proof text. Example for proof text. Example for proof text. Example for proof text. Example for proof text. Example for proof text. Example for proof text. Example for proof text. Example for proof text.
\end{proof}

\begin{algorithm}
\caption{\enskip Pseudocode for our algorithm}\label{alg1}
\begin{algorithmic}
  \For each frame
  \For water particles $f_{i}$
  \State compute fluid flow \cite{Hirt1974}
  \State compute fluid--solid interaction \cite{Benson1992}
  \State apply adhesion and surface tension \cite{Margolin2003}
  \EndFor
   \For solid particles $s_{i}$
   \For neighboring water particles $f_{j}$
   \State compute virtual water film \\(see Section~\ref{sec3})
   \EndFor
   \EndFor
   \For solid particles $s_{i}$
   \For neighboring water particles $f_{j}$
   \State compute growth direction vector \\(see Section~\ref{sec2})
   \EndFor
   \EndFor
   \For solid particles $s_{i}$
   \For neighboring water particles $f_{j}$
   \State compute $F_{\theta}$ (see Section~\ref{sec1})
   \State compute $CE(s_{i},f_{j})$ \\(see Section~\ref{sec3})
   \If $CE(b_{i}, f_{j})$ $>$ glaze threshold
   \State $j$th water particle's phase $\Leftarrow$ ICE
   \EndIf
   \If $CE(c_{i}, f_{j})$ $>$ icicle threshold
   \State $j$th water particle's phase $\Leftarrow$ ICE
   \EndIf
   \EndFor
   \EndFor
  \EndFor
\end{algorithmic}
\end{algorithm}



\section{Conclusions}\label{sec5}

Lorem ipsum dolor sit amet, consectetuer adipiscing elit. Ut purus elit, vestibulum ut, placerat ac, adipiscing vitae,
felis. Curabitur dictum gravida mauris. Nam arcu libero, nonummy eget, consectetuer id, vulputate a, magna. Donec
vehicula augue eu neque. Pellentesque habitant morbi tristique senectus et netus et malesuada fames ac turpis egestas.
Mauris ut leo. Cras viverra metus rhoncus sem. Nulla et lectus vestibulum urna fringilla ultrices. Phasellus eu tellus
sit amet tortor gravida placerat. Integer sapien est, iaculis in, pretium quis, viverra ac, nunc. Praesent eget sem vel
leo ultrices bibendum. Aenean faucibus. Morbi dolor nulla, malesuada eu, pulvinar at, mollis ac, nulla. Curabitur
auctor semper nulla. Donec varius orci eget risus. Duis nibh mi, congue eu, accumsan eleifend, sagittis quis, diam.
Duis eget orci sit amet orci dignissim rutrum.


%\backmatter
\bmsection*{Author contributions}

This is an author contribution text. This is an author contribution text. This is an author contribution text. This is an author contribution text. This is an author contribution text.

\bmsection*{Acknowledgments}
This is acknowledgment text. \cite{Kenamond2013} Provide text here. This is acknowledgment text. Provide text here. This is acknowledgment text. Provide text here. This is acknowledgment text. Provide text here. This is acknowledgment text. Provide text here. This is acknowledgment text. Provide text here. This is acknowledgment text. Provide text here. This is acknowledgment text. Provide text here. This is acknowledgment text. Provide text here.


\bmsection*{Financial disclosure}

None reported.

\bmsection*{Conflict of interest}

The authors declare no potential conflict of interests.


\bibliography{paper/refs/competing-risk}


\bmsection*{Supporting information}

Additional supporting information may be found in the
online version of the article at the publisher’s website.




\appendix

\bmsection{Program codes appear in Appendix\label{app1}}
\vspace*{12pt}
Using the package {\tt listings} you can add non-formatted text as you would do with \verb|\begin{verbatim}| but its main aim is to include the source code of any programming language within your document.\newline Use \verb|\begin{lstlisting}...\end{lstlisting}| for program codes without mathematics.

The {\tt listings} package supports all the most common languages and it is highly customizable. If you just want to write code within your document, the package provides the {\tt lstlisting} environment; the output will be in Computer Modern typewriter font. Refer to the below example:


\begin{lstlisting}[caption={Descriptive caption text},label=DescriptiveLabel, basicstyle=\fontsize{8}{10}\selectfont\ttfamily]
for i:=maxint to 0 do
begin
{ do nothing }
end;
Write('Case insensitive ');
WritE('Pascal keywords.');
\end{lstlisting}



\bmsubsection{Subsection title of first appendix\label{app1.1a}}

Nam dui ligula, fringilla a, euismod sodales, sollicitudin vel, wisi. Morbi auctor lorem non justo. Nam lacus libero,
pretium at, lobortis vitae, ultricies et, tellus. Donec aliquet, tortor sed accumsan bibendum, erat ligula aliquet magna,
vitae ornare odio metus a mi. Morbi ac orci et nisl hendrerit mollis. Suspendisse ut massa. Cras nec ante. Pellentesque
a nulla. Cum sociis natoque penatibus et magnis dis parturient montes, nascetur ridiculus mus. Aliquam tincidunt
urna. Nulla ullamcorper vestibulum turpis. Pellentesque cursus luctus mauris.

Nulla malesuada porttitor diam. Donec felis erat, congue non, volutpat at, tincidunt tristique, libero. Vivamus
viverra fermentum felis. Donec nonummy pellentesque ante. Phasellus adipiscing semper elit. Proin fermentum massa
ac quam. Sed diam turpis, molestie vitae, placerat a, molestie nec, leo. Maecenas lacinia. Nam ipsum ligula, eleifend
at, accumsan nec, suscipit a, ipsum. Morbi blandit ligula feugiat magna. Nunc eleifend consequat lorem. Sed lacinia
nulla vitae enim. Pellentesque tincidunt purus vel magna. Integer non enim. Praesent euismod nunc eu purus. Donec
bibendum quam in tellus. Nullam cursus pulvinar lectus. Donec et mi. Nam vulputate metus eu enim. Vestibulum
pellentesque felis eu massa.
Nulla malesuada porttitor diam. Donec felis erat, congue non, volutpat at, tincidunt tristique, libero. Vivamus
viverra fermentum felis. Donec nonummy pellentesque ante. Phasellus adipiscing semper elit. Proin fermentum massa
ac quam. Sed diam turpis, molestie vitae, placerat a, molestie nec, leo. Maecenas lacinia. Nam ipsum ligula, eleifend
at, accumsan nec, suscipit a, ipsum. Morbi blandit ligula feugiat magna. Nunc eleifend consequat lorem. Sed lacinia
nulla vitae enim. Pellentesque tincidunt purus vel magna. Integer non enim. Praesent euismod nunc eu purus. Donec
bibendum quam in tellus. Nullam cursus pulvinar lectus. Donec et mi. Nam vulputate metus eu enim. Vestibulum
pellentesque felis eu massa.

\bmsubsubsection{Subsection title of first appendix\label{app1.1.1a}}

\noindent\textbf{Unnumbered figure}


\begin{center}
\includegraphics[width=7pc,height=8pc,draft]{empty}
\end{center}


Fusce mauris. Vestibulum luctus nibh at lectus. Sed bibendum, nulla a faucibus semper, leo velit ultricies tellus, ac
venenatis arcu wisi vel nisl. Vestibulum diam. Aliquam pellentesque, augue quis sagittis posuere, turpis lacus congue
quam, in hendrerit risus eros eget felis. Maecenas eget erat in sapien mattis porttitor. Vestibulum porttitor. Nulla
facilisi. Sed a turpis eu lacus commodo facilisis. Morbi fringilla, wisi in dignissim interdum, justo lectus sagittis dui, et
vehicula libero dui cursus dui. Mauris tempor ligula sed lacus. Duis cursus enim ut augue. Cras ac magna. Cras nulla.

Nulla egestas. Curabitur a leo. Quisque egestas wisi eget nunc. Nam feugiat lacus vel est. Curabitur consectetuer.
Suspendisse vel felis. Ut lorem lorem, interdum eu, tincidunt sit amet, laoreet vitae, arcu. Aenean faucibus pede eu
ante. Praesent enim elit, rutrum at, molestie non, nonummy vel, nisl. Ut lectus eros, malesuada sit amet, fermentum
eu, sodales cursus, magna. Donec eu purus. Quisque vehicula, urna sed ultricies auctor, pede lorem egestas dui, et
convallis elit erat sed nulla. Donec luctus. Curabitur et nunc. Aliquam dolor odio, commodo pretium, ultricies non,
pharetra in, velit. Integer arcu est, nonummy in, fermentum faucibus, egestas vel, odio.

\bmsection{Section title of second appendix\label{app2}}%
\vspace*{12pt}
Fusce mauris. Vestibulum luctus nibh at lectus. Sed bibendum, nulla a faucibus semper, leo velit ultricies tellus, ac
venenatis arcu wisi vel nisl. Vestibulum diam. Aliquam pellentesque, augue quis sagittis posuere, turpis lacus congue
quam, in hendrerit risus eros eget felis. Maecenas eget erat in sapien mattis porttitor. Vestibulum porttitor. Nulla
facilisi. Sed a turpis eu lacus commodo facilisis. Morbi fringilla, wisi in dignissim interdum, justo lectus sagittis dui, et
vehicula libero dui cursus dui. Mauris tempor ligula sed lacus. Duis cursus enim ut augue. Cras ac magna. Cras nulla (Figure~\ref{fig5}).

Nulla egestas. Curabitur a leo. Quisque egestas wisi eget nunc. Nam feugiat lacus vel est. Curabitur consectetuer.
Suspendisse vel felis. Ut lorem lorem, interdum eu, tincidunt sit amet, laoreet vitae, arcu. Aenean faucibus pede eu
ante. Praesent enim elit, rutrum at, molestie non, nonummy vel, nisl. Ut lectus eros, malesuada sit amet, fermentum
eu, sodales cursus, magna. Donec eu purus. Quisque vehicula, urna sed ultricies auctor, pede lorem egestas dui, et
convallis elit erat sed nulla. Donec luctus. Curabitur et nunc. Aliquam dolor odio, commodo pretium, ultricies non,
pharetra in, velit. Integer arcu est, nonummy in, fermentum faucibus, egestas vel, odio.

%== Figure 4 ==
%% Example for figure inside appendix
\begin{figure}[b]
\centerline{\includegraphics[height=10pc,width=78mm,draft]{empty}}
\caption{This is an example for appendix figure.\label{fig5}}
\end{figure}

\bmsubsection{Subsection title of second appendix\label{app2.1a}}

Sed commodo posuere pede. Mauris ut est. Ut quis purus. Sed ac odio. Sed vehicula hendrerit sem. Duis non odio.
Morbi ut dui. Sed accumsan risus eget odio. In hac habitasse platea dictumst. Pellentesque non elit. Fusce sed justo
eu urna porta tincidunt. Mauris felis odio, sollicitudin sed, volutpat a, ornare ac, erat. Morbi quis dolor. Donec
pellentesque, erat ac sagittis semper, nunc dui lobortis purus, quis congue purus metus ultricies tellus. Proin et quam.
Class aptent taciti sociosqu ad litora torquent per conubia nostra, per inceptos hymenaeos. Praesent sapien turpis,
fermentum vel, eleifend faucibus, vehicula eu, lacus.

Pellentesque habitant morbi tristique senectus et netus et malesuada fames ac turpis egestas. Donec odio elit,
dictum in, hendrerit sit amet, egestas sed, leo. Praesent feugiat sapien aliquet odio. Integer vitae justo. Aliquam
vestibulum fringilla lorem. Sed neque lectus, consectetuer at, consectetuer sed, eleifend ac, lectus. Nulla facilisi.
Pellentesque eget lectus. Proin eu metus. Sed porttitor. In hac habitasse platea dictumst. Suspendisse eu lectus. Ut
mi mi, lacinia sit amet, placerat et, mollis vitae, dui. Sed ante tellus, tristique ut, iaculis eu, malesuada ac, dui.
Mauris nibh leo, facilisis non, adipiscing quis, ultrices a, dui.

\bmsubsubsection{Subsection title of second appendix\label{app2.1.1a}}

Lorem ipsum dolor sit amet, consectetuer adipiscing elit. Ut purus elit, vestibulum ut, placerat ac, adipiscing vitae,
felis. Curabitur dictum gravida mauris. Nam arcu libero, nonummy eget, consectetuer id, vulputate a, magna. Donec
vehicula augue eu neque. Pellentesque habitant morbi tristique senectus et netus et malesuada fames ac turpis egestas.
Mauris ut leo. Cras viverra metus rhoncus sem. Nulla et lectus vestibulum urna fringilla ultrices. Phasellus eu tellus
sit amet tortor gravida placerat. Integer sapien est, iaculis in, pretium quis, viverra ac, nunc. Praesent eget sem vel
leo ultrices bibendum. Aenean faucibus. Morbi dolor nulla, malesuada eu, pulvinar at, mollis ac, nulla. Curabitur
auctor semper nulla. Donec varius orci eget risus. Duis nibh mi, congue eu, accumsan eleifend, sagittis quis, diam.
Duis eget orci sit amet orci dignissim rutrum (Table~\ref{tab4}).

Nam dui ligula, fringilla a, euismod sodales, sollicitudin vel, wisi. Morbi auctor lorem non justo. Nam lacus libero,
pretium at, lobortis vitae, ultricies et, tellus. Donec aliquet, tortor sed accumsan bibendum, erat ligula aliquet magna,
vitae ornare odio metus a mi. Morbi ac orci et nisl hendrerit mollis. Suspendisse ut massa. Cras nec ante. Pellentesque
a nulla. Cum sociis natoque penatibus et magnis dis parturient montes, nascetur ridiculus mus. Aliquam tincidunt
urna. Nulla ullamcorper vestibulum turpis. Pellentesque cursus luctus mauris.

\begin{table*}[t]%
\centering
\caption{This is an example of Appendix table showing food requirements of army, navy and airforce.\label{tab4}}%
\begin{tabular*}{\textwidth}{@{\extracolsep\fill}llllll@{\extracolsep\fill}}%
\toprule
\textbf{Col1 head} & \textbf{Col2 head} & \textbf{Col3 head} & \textbf{Col4 head} & \textbf{Col5 head} & \textbf{Col6 head} \\
\midrule
col1 text & col2 text & col3 text & col4 text & col5 text & col6 text\\
col1 text & col2 text & col3 text & col4 text & col5 text & col6 text\\
col1 text & col2 text & col3 text& col4 text & col5 text & col6 text\\
\bottomrule
\end{tabular*}
\end{table*}


Example for an equation inside appendix
\begin{equation}
{\mathcal{L}} = i \bar{\psi} \gamma^\mu D_\mu \psi - \frac{1}{4} F_{\mu\nu}^a F^{a\mu\nu} - m \bar{\psi} \psi\label{eq25}
\end{equation}

\bmsection{Example of another appendix section\label{app3}}%
\vspace*{12pt}
This is sample for paragraph text this is sample for paragraph text  this is sample for paragraph text this is sample for paragraph text this is sample for paragraph text this is sample for paragraph text this is sample for paragraph text this is sample for paragraph text this is sample for paragraph text this is sample for paragraph text this is sample for paragraph text this is sample for paragraph text this is sample for paragraph text this is sample for paragraph text this is sample for paragraph text this is sample for paragraph text this is sample for paragraph text this is sample for paragraph text this is sample for paragraph text this is sample for paragraph text this is sample for paragraph text this is sample for paragraph text this is sample for paragraph text this is sample for paragraph text this is sample for paragraph text this is sample for paragraph text this is sample for paragraph text this is sample for paragraph text this is sample for paragraph text this is sample for paragraph text this is sample for paragraph text this is sample for paragraph text this is sample for paragraph text



Nam dui ligula, fringilla a, euismod sodales, sollicitudin vel, wisi. Morbi auctor lorem non justo. Nam lacus libero,
pretium at, lobortis vitae, ultricies et, tellus. Donec aliquet, tortor sed accumsan bibendum, erat ligula aliquet magna,
vitae ornare odio metus a mi. Morbi ac orci et nisl hendrerit mollis. Suspendisse ut massa. Cras nec ante. Pellentesque
a nulla. Cum sociis natoque penatibus et magnis dis parturient montes, nascetur ridiculus mus. Aliquam tincidunt
urna. Nulla ullamcorper vestibulum turpis. Pellentesque cursus luctus mauris.

Nulla malesuada porttitor diam. Donec felis erat, congue non, volutpat at, tincidunt tristique, libero. Vivamus
viverra fermentum felis. Donec nonummy pellentesque ante. Phasellus adipiscing semper elit. Proin fermentum massa
ac quam. Sed diam turpis, molestie vitae, placerat a, molestie nec, leo. Maecenas lacinia. Nam ipsum ligula, eleifend
at, accumsan nec, suscipit a, ipsum. Morbi blandit ligula feugiat magna. Nunc eleifend consequat lorem. Sed lacinia
nulla vitae enim. Pellentesque tincidunt purus vel magna. Integer non enim. Praesent euismod nunc eu purus. Donec
bibendum quam in tellus. Nullam cursus pulvinar lectus. Donec et mi. Nam vulputate metus eu enim. Vestibulum
pellentesque felis eu massa.
\begin{equation}
\mathcal{L} = i \bar{\psi} \gamma^\mu D_\mu \psi
    - \frac{1}{4} F_{\mu\nu}^a F^{a\mu\nu} - m \bar{\psi} \psi
\label{eq26}
\end{equation}

Nulla malesuada porttitor diam. Donec felis erat, congue non, volutpat at, tincidunt tristique, libero. Vivamus
viverra fermentum felis. Donec nonummy pellentesque ante. Phasellus adipiscing semper elit. Proin fermentum massa
ac quam. Sed diam turpis, molestie vitae, placerat a, molestie nec, leo. Maecenas lacinia. Nam ipsum ligula, eleifend
at, accumsan nec, suscipit a, ipsum. Morbi blandit ligula feugiat magna. Nunc eleifend consequat lorem. Sed lacinia
nulla vitae enim. Pellentesque tincidunt purus vel magna. Integer non enim. Praesent euismod nunc eu purus. Donec
bibendum quam in tellus. Nullam cursus pulvinar lectus. Donec et mi. Nam vulputate metus eu enim. Vestibulum
pellentesque felis eu massa.

Quisque ullamcorper placerat ipsum. Cras nibh. Morbi vel justo vitae lacus tincidunt ultrices. Lorem ipsum dolor sit
amet, consectetuer adipiscing elit. In hac habitasse platea dictumst. Integer tempus convallis augue. Etiam facilisis.
Nunc elementum fermentum wisi. Aenean placerat. Ut imperdiet, enim sed gravida sollicitudin, felis odio placerat
quam, ac pulvinar elit purus eget enim. Nunc vitae tortor. Proin tempus nibh sit amet nisl. Vivamus quis tortor
vitae risus porta vehicula.


\begin{center}
\begin{tabular*}{250pt}{@{\extracolsep\fill}lcc@{\extracolsep\fill}}%
\toprule
\textbf{Col1 head} & \textbf{Col2 head} & \textbf{Col3 head} \\
\midrule
col1 text & col2 text & col3 text \\
col1 text & col2 text & col3 text \\
col1 text & col2 text & col3 text \\
\bottomrule
\end{tabular*}
\end{center}


Quisque ullamcorper placerat ipsum. Cras nibh. Morbi vel justo vitae lacus tincidunt ultrices. Lorem ipsum dolor sit
amet, consectetuer adipiscing elit. In hac habitasse platea dictumst. Integer tempus convallis augue. Etiam facilisis.
Nunc elementum fermentum wisi. Aenean placerat. Ut imperdiet, enim sed gravida sollicitudin, felis odio placerat
quam, ac pulvinar elit purus eget enim. Nunc vitae tortor. Proin tempus nibh sit amet nisl. Vivamus quis tortor
vitae risus porta vehicula.

Fusce mauris. Vestibulum luctus nibh at lectus. Sed bibendum, nulla a faucibus semper, leo velit ultricies tellus, ac
venenatis arcu wisi vel nisl. Vestibulum diam. Aliquam pellentesque, augue quis sagittis posuere, turpis lacus congue
quam, in hendrerit risus eros eget felis. Maecenas eget erat in sapien mattis porttitor. Vestibulum porttitor. Nulla
facilisi. Sed a turpis eu lacus commodo facilisis. Morbi fringilla, wisi in dignissim interdum, justo lectus sagittis dui, evehicula libero dui cursus dui. Mauris tempor ligula sed lacus. Duis cursus enim ut augue. Cras ac magna. Cras nulla.
Nulla egestas. Curabitur a leo. Quisque egestas wisi eget nunc. Nam feugiat lacus vel est. Curabitur consectetuer.

Pellentesque habitant morbi tristique senectus et netus et malesuada fames ac turpis egestas. Donec odio elit,
dictum in, hendrerit sit amet, egestas sed, leo. Praesent feugiat sapien aliquet odio. Integer vitae justo. Aliquam
vestibulum fringilla lorem. Sed neque lectus, consectetuer at, consectetuer sed, eleifend ac, lectus. Nulla facilisi.
Pellentesque eget lectus. Proin eu metus. Sed porttitor. In hac habitasse platea dictumst. Suspendisse eu lectus. Ut
mi mi, lacinia sit amet, placerat et, mollis vitae, dui. Sed ante tellus, tristique ut, iaculis eu, malesuada ac, dui.
Mauris nibh leo, facilisis non, adipiscing quis, ultrices a, dui.

%%\nocite{*}% Show all bib entries - both cited and uncited; comment this line to view only cited bib entries;


\end{document}
